\documentclass[10pt,a4paper,uplatex]{jsarticle}
\usepackage{bm}
\usepackage{graphicx}
\usepackage[truedimen,left=25truemm,right=25truemm,top=25truemm,bottom=25truemm]{geometry}
\usepackage{jpdoc}


\def\title{定款}

\newcounter{NumOfMembers}\setcounter{NumOfMembers}{1}
\newcounter{DocumentsPerMember}\setcounter{DocumentsPerMember}{1}
\def\Name#1{\ifcase#1\or
株式会社ゼロアイデア\or%甲
\else\fi}

\def\Address#1{\ifcase#1\or
京都府福知山市三和町大身1305\or%甲
\else\fi}

\def\Representative#1{\ifcase#1\or
代表取締役 大槻祥平\or%甲
\else\fi}

\def\BlankSignature#1{\ifcase#1\or
0\or%甲
0\else0\fi}

\begin{document}
\newpage
{\centering \Large\bf \title  \vskip 0em}
\vskip 2em
\subsection{総則}
\article{商号}
当会社は、株式会社ゼロアイデアと称し、英文では Zeroidea Co., Ltd. と表記する。

当会社は、次の事業を営むことを目的とする。
\begin{enumerate}
  \item[1] 暗号通貨を中心とした電子的資産の国内外取引所における自己資金による資産運用
  \item[2] 特にハードウェアに関わるオープンソースの枠組みへの参画と、それによる価値、産業サイクルの創出
  \item[3] 分野横断的な技術と社会課題の探求による高い利益率を追求した事業モデルの創出と、そのための調査・研究・開発
  \item[4] 上記(1)(2)(3)のためのソフトウェアの開発、運用、買取、保守及び輸出入
  \item[5] 上記(1)(2)(3)のためのハードウェアの設計、開発、販売、買取、保守及び輸出入
  \item[6] その他適法な一切の事業
\end{enumerate}
\article{本店の所在地}
当会社は、本店を京都府福知山市に置く。
\article{公告の方法}
当会社の公告は、電子公告によって行う。ただし、事故その他やむを得ない事由によって電子公告による公告をすることができない場合は、官報に掲載してする。


\subsection{株式}
\article{発行可能株式総数}
当会社の発行可能株式総数は、10万株とする。
\article{株券の不発行}
当会社の株式については、株券を発行しない。
\article{株式の譲渡制限}
当会社の発行する株式の譲渡による取得については、取締役の承認を受けなければならない。ただし、当会社の株主に譲渡する場合には、承認をしたものとみなす。
\article{相続人等に対する株式の売渡し請求}
当会社は、相続、合併その他の一般承継により当会社の株式を取得した者に対し、当該株式を当会社に売り渡すことを請求することができる。
\article{株主名簿記載事項の記載の請求}
当会社の株式取得者が、株主名簿記載事項を株主名簿に記載又は記録することを請求するには、当会社所定の書式による請求書に、株式取得者とその取得した株式の株主として株主名簿に記載若しくは記録された者又はその相続人その他の一般承継人が署名又は記名押印し、共同して請求しなければならない。ただし、法務省令の定める事由による場合は、株式取得者が単独で上記請求をすることができる。
\term 前項の規定にかかわらず、利害関係人の利益を害するおそれがないものとして法務省令に定める場合には、株式取得者が単独で株主名簿記載事項を株主名簿に記載又は記録することを請求することができる。
\article{質権の登録及び信託財産の表示}
当会社の発行する株式につき、質権の登録又は信託財産の表示を請求するには、当会社所定の書式による請求書に当事者が署名又は記名押印し、これを当会社に提出しなければならない。その登録又は表示の抹消についても同様とする。
\article{手数料}
前二条に定める請求をする場合には、当会社所定の手数料を支払わなければならない。
\article{基準日}
当会社は、毎事業年度末日の最終の株主名簿に記載又は記録された議決権を有する株主(以下「基準日株主」という。)をもってその事業年度に関する定時株主総会において権利を行使することができる株主とする。但し、当該基準日株主の権利を害しない場合には、当会社は基準日後に、募集株式の発行等、吸収合併、株式交換又は吸収分割等により株式を取得した者の全部又は一部を、当該定時株主総会において権利を行使することができる株主と定めることができる。
\article{株主の住所等の届出}
当会社の株主及び登録株式質権者又はその法定代理人若しくは代表者は、当会社所定の書式により、住所、氏名又は名称及び印鑑を当会社に届け出なければならない。
\term
前項の届出事項を変更したときも同様とする。


\subsection{株主総会}
\article{招集}
当会社の定時株主総会は、毎事業年度の終了後3か月以内にこれを招集し、臨時株主総会は、必要がある場合にこれを招集する。
\article{招集手続きの省略}
株主総会を招集するには、会日の1週間前までに、書面投票又は電子投票を認める場合は2週間前までに、議決権を行使することができる各株主に対して招集通知を発するものとする。 ただし、議決権を行使することができる株主全員の同意があるときは、書面投票又は電子投票を認める場合を除き、招集の手続を経ることなく開催することができる。
\article{招集権者及び議長}
株主総会は法令に別段の定めがある場合を除き、代表取締役社長がこれを招集し、議長となる。
\term 代表取締役に事故があるときは、他の取締役が議長になる。
\term 取締役全員に事故があるときは、総会において出席株主のうちから議長を選出する。
\article{決議の方法}
株主総会の決議は、法令又は定款に別段の定めがある場合を除き、議決権を行使することができる株主の議決権の過半数を有する株主が出席し、出席した当該株主の議決権の過半数をもって行う。
\term 会社法第309条第2項に定める決議は、当該株主総会において議決権を行使することができる株主の議決権の過半数を有する株主が出席し、その議決権の3分の2以上をもって決する。
\article{株主総会の決議の省略}
株主総会の決議の目的たる事項について、取締役又は株主から提案があった場合において、その事項につき議決権を行使することができるすべての株主が、書面によってその提案に同意したときは、その提案を可決する旨の株主総会の決議があったものとみなす。
\article{議決権の代理行使}
株主又はその法定代理人は、当会社の議決権を有する他の株主を代理人として議決権を行使することができる。但し、この場合には、株主又は代理人は代理権を証明する書面を、株主総会ごとに当会社に提出しなければならない。
\article{株主総会議事録}
株主総会の議事については、法令に定める事項を記載した議事録を作成し、取締役社長又は取締役会が別途指定する取締役がこれに署名、記名押印又は電子署名する。


\subsection{取締役及び代表取締役}
\article{取締役の員数}
当会社の取締役は、1名以上とする。
\article{取締役の選任の方法}
取締役の選任は、株主総会において議決権を行使することができる株主の議決権の過半数を有する株主が出席し、出席した当該株主の議決権の過半数をもって行う。
\term 取締役の選任決議は、累積投票によらないものとする。
\article{取締役の解任の方法}
取締役の解任は、株主総会において議決権を行使することができる株主の議決権の過半数を有する株主が出席し、出席した当該株主の議決権の過半数をもって行う。
\article{取締役の任期}
取締役の任期は、選任後10年以内に終了する事業年度のうち最終のものに関する定時株主総会の終結の時までとする。
\term 任期満了前に退任した取締役の補欠として、又は増員により選任された取締役の任期は、前任者又は他の在任取締役の任期の残存期間と同一とする。
\article{代表取締役及び役付取締役}
当会社に取締役を2名以上置く場合には、取締役の互選により代表取締役1名以上を定める。
\term 当会社に置く取締役が1名の場合には、その取締役を代表取締役とする。
\term 代表取締役は当会社の業務を執行する。
\term 会社は、代表取締役の決議によって、取締役社長1名のほか、必要に応じて、CEO、会長、副社長、専務取締役、常務取締役各若干名を選定することができる。
\article{報酬等}
取締役の報酬、賞与その他の職務執行の対価として当会社から受ける財産上の利益(以下「報酬等」という。)は、株主総会の決議によって定める。

\subsection{計算}
\article{事業年度}
当会社の事業年度は、毎年6月1日から翌年5月31日までとする。
\article{配当金}
剰余金の配当は、毎事業年度末日現在の最終の株主名簿に記載又は記録された株主又は登録株式質権者に対して行う。
\term 前項のほか、基準日を定めて剰余金の配当をすることができる。
\term 前二項の配当金がその交付開始の日から満3年を経過してもなお受領されないときは、当会社はその支払義務を免れる。
\term 前項の期間内において、第1項および第2項の配当金には利息を付さない。

\subsection{附則}
\article{定款に定めのない事項}
本定款に定めのない事項については、すべて会社法その他の法令の定めるところによる。


%CERTIFICATION FIELD%

\begin{flushleft}
\today\\
\vspace{10pt}
\MakeSignatureField
\end{flushleft}
%CERTIFICATION FIELD%
\end{document}